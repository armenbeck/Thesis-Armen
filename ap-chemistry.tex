\ProvidesFile{ap-chemistry.tex}[2021-08-23 chemistry appendix]

\begin{VerbatimOut}{z.out}
\chapter{CHEMISTRY}
\label{ch:chemistry}

\ix{chemistry}
\ix{Chemistry appendix}

\end{VerbatimOut}

\MyIO


\begin{VerbatimOut}{z.out}


\section{Chemical Diagrams}

The chemplants package \cite{feffin2019}
extends the
\href{http://ctan.math.washington.edu/tex-archive/graphics/pgf/base/doc/pgfmanual.pdf}{\TikZLogo} package
to draw chemical process units.
\end{VerbatimOut}

\MyIO


\begin{VerbatimOut}{z.out}


\section{Chemical Equations}

The mhchem Bundle \cite{hensel2018}
contains mhchem v4.08 (chemical equations),
hpstatement v1.02 (official hazard and precautionary statements),
and rsphrase v3.11 (official rist and safety phrases).
\end{VerbatimOut}

\MyIO


\begin{VerbatimOut}{z.out}

Defined in thesis.tex: \nitrate.
\end{VerbatimOut}

\MyIO


\begin{VerbatimOut}{z.out}

% See page 1 of
%     https://www.thoughtco.com/what-is-a-chemical-equation-604026
\ce{CH4 + 2O2 -> CO2 + 2H2O}
\end{VerbatimOut}

\MyIO


\begin{VerbatimOut}{z.out}

% See page 1 of
%     https://www.thoughtco.com/what-is-a-chemical-equation-604026
\ce{2H2(g) + O2(g) -> 2H2O(l)}
\end{VerbatimOut}

\MyIO


\begin{VerbatimOut}{z.out}

% See page 1 of
% https://www.thoughtco.com/definition-of-ionic-equation-605262
\ce{Ag+(aq) + NO3-(aq) + Na+(aq) + Cl-(aq) -> AgCl(s) + Na+(aq) + NO3-(aq)}
is an ionic equation of the chemical reaction:
\ce{AgNO3(aq) + NaCl(aq) -> AgCl(s) + NaNO3(aq)}
\end{VerbatimOut}

\MyIO


\begin{VerbatimOut}{z.out}

% See page 1 of
%     https://www.thoughtco.com/definition-of-balanced-equation-and-examples-604380
\ce{Fe2O3 + C -> Fe + CO2}
\end{VerbatimOut}

\MyIO


\begin{VerbatimOut}{z.out}

% From page 1 of
%     https://www.thoughtco.com/definition-of-molecular-equation-605366

For example, in the reaction between sodium chloride
(\ce{NaCl})
and silver nitrate
(\ce{AgNO3}),
the molecular reaction is:

\ce{NaCl(aq) + AgNO3 -> NaNO3(aq) + AgCl(s)}
\end{VerbatimOut}

\MyIO


\begin{VerbatimOut}{z.out}

The complete ionic equation is:

\ce{Na+(aq) + Cl-(aq) + Ag+(aq) + NO3-(aq) -> AgCl(s) + Na+(aq) + NO3-(aq)}
\end{VerbatimOut}

\MyIO


\begin{VerbatimOut}{z.out}

Ruben Meerman \cite[starting at 5:25]{meerman} claims this equation
\begin{center}
  \ce{C55H104O6 + 78O2 -> 55CO2 + 52H2O + energy}\endgraf
\end{center}
describes weight loss.
\end{VerbatimOut}

\MyIO


\begin{VerbatimOut}{z.out}

And with better annotation:

\begin{center}
  \newcommand{\vph}{{\vphantom{\large Ag}}}
  \ce{
    $\underset{\text{\vph \footnotesize human fat}}{\ce{C55H104O6}}$
    +
    $\underset{\text{\vph \footnotesize oxygen}}{\ce{78O2}}$
    ->
    $\underset{\text{\vph \footnotesize carbon dioxide}}{\ce{55CO2}}$
    +
    $\underset{\text{\vph \footnotesize water}}{\ce{52H2O}}$
    +
    $\underset{\text{\vph \footnotesize body heat, moving, thinking, growing}}{\text{energy}}$
  }
\end{center}
\end{VerbatimOut}

\MyIO


\begin{VerbatimOut}{z.out}

And with still better annotation:

\begin{center}
  \newcommand{\Fs}{\scriptsize}
  \begin{tabular}{@{}c@{}c@{}c@{}c@{}c@{}c@{}c@{}c@{}c@{}}
    &                                                       %  1. C55H10406
      &                                                     %  2. +
      &                                                     %  3. 78O2
      &                                                     %  4. ->
      &                                                     %  5. 55CO2
      &                                                     %  6. +
      &                                                     %  7. 52H2O
      &                                                     %  8. +
      \Fs calories\\                                        %  9. energy
    %
    \Fs kg&                                                 %  1.
      &                                                     %  2.
      \Fs kg&                                               %  3.
      &                                                     %  4.
      \Fs kg&                                               %  5.
      &                                                     %  6.
      \Fs kg&                                               %  7.
      &                                                     %  8.
      \Fs kJ\\                                              %  9.
    %
    \noalign{\vspace{3pt}}
    %
    \ce{C55H104O6}                                          %  1.
      & \ce{+}                                              %  2.
      & \ce{78O2}                                           %  3.
      & \ce{->}                                             %  4.
      & \ce{55CO2}                                          %  5.
      & \ce{+}                                              %  6.
      & \ce{52H2O}                                          %  7.
      & \ce{+}                                              %  8.
      & energy\\                                            %  9.
    %
    \Fs human fat&                                          %  1.
      &                                                     %  2.
      \Fs oxygen&                                           %  3.
      &                                                     %  4.
      \Fs carbon dioxide&                                   %  5.
      &                                                     %  6.
      \Fs water&                                            %  7.
      &                                                     %  8.
      \Fs body heat, moving, thinking, growing\\            %  9.
  \end{tabular}
\end{center}
\end{VerbatimOut}

\MyIO


\begin{VerbatimOut}{z.out}


\section{Chemical Figures}

Below is an example of how to use the chemfig package \cite{tellechea2021}.

% Chicago Manual of Style Online, 17 edition, section 9.61 states
% that 72--73, not 72--3, should be used.
Here is the chemical figure
for Penicillin \cite[pages~72--73]{tellechea2021}:\\

\chemfig{
  [:-90]HN(-[::-45](-[::-45]R)=[::+45]O)>[::+45]*4(-(=O)-N*5(-(<:(=[::-60]O)
  -[::+60]OH)-(<[::+0])(<:[::-108])-S>)--)
}
\end{VerbatimOut}

\MyIO


\begin{VerbatimOut}{z.out}
\newpage
\section{Chemical Schemes}

Below are some examples of how to do schemes.

\begin{scheme}[ht]
  \caption{This is the first scheme caption.}
  \vspace*{6pt}
  \begin{center}
    This is the first scheme.
  \end{center}
\end{scheme}
\end{VerbatimOut}

\MyIO


\begin{VerbatimOut}{z.out}
\begin{scheme}[ht]
  \caption{This is the second scheme caption.}
  \vspace*{6pt}
  \begin{center}
    % Next line was added to make scheme a little smaller.
    \scriptsize\setchemfig{bond offset=1pt,atom sep=3em,compound sep=6em}
    \schemestart
      \chemfig{-[:30](-[2])-[:-30]OH}
      \arrow
      \chemfig{-[:30](-[2])=^[:-30]O}
    \schemestop
  \end{center}
\end{scheme}
\end{VerbatimOut}

\MyIO


\begin{VerbatimOut}{z.out}
\newpage
\begin{scheme}[ht]
  \caption{%
    The Fischer indole synthesis
    \cite[pages~74--75]{tellechea2021}.%
  }
  \vspace*{6pt}
  \begin{center}
    % Next line was added to make scheme a little smaller.
    \scriptsize\setchemfig{bond offset=1pt,atom sep=3em,compound sep=6em}
    \schemestart
      \chemfig{*6(=-*6(-\chembelow{N}{H}-NH_2)=-=-)}
      \+
      \chemfig{(=[:-150]O)(-[:-30]R_2)-[2]-[:150]R_1}
      \arrow(.mid east--.mid west){->[\chemfig{H^+}]}
      \chemfig{*6(-=*5(-\chembelow{N}{H}-(-R_2)=(-R_1)-)-=-=)}
    \schemestop
  \end{center}
\end{scheme}
\end{VerbatimOut}

\MyIO


\begin{VerbatimOut}{z.out}
\newpage
\begin{scheme}[ht]
  \caption{%
    The Cannizzaro reaction
    \cite[pages~77--78]{tellechea2021}.%
  }
  \vspace*{12pt}
  \begin{center}
    % Next line was added to make scheme a little smaller.
    \scriptsize\setchemfig{bond offset=1pt,atom sep=3em,compound sep=6em}
    \schemestart
      \chemfig{[:-30]*6(=-=(-@{atoc}C([6]=[@{db}]@{atoo1}O)-H)-=-)}
      \arrow(start.mid east--.mid west){->[\chemfig{@{atoo2}\chemabove{O}{\scriptstyle\ominus}}H]}
      \chemmove[-stealth,shorten >=2pt,dash pattern=on 1pt off 1pt,thin]{
        \draw[shorten <=8pt](atoo2) ..controls +(up:10mm) and +(up:10mm)..(atoc);
        \draw[shorten <=2pt](db) ..controls +(left:5mm) and +(west:5mm)..(atoo1);}
      \chemfig{[:-30]*6(=-=(-C([6]-[@{sb1}]@{atoo1}\chembelow{O}{\scriptstyle\ominus})
        ([2]-OH)-[@{sb2}]H)-=-)}
      \hspace{1cm}
      \chemfig{[:-30]*6((-@{atoc}C([6]=[@{db}]@{atoo2}O)-[2]H)-=-=-=)}
      \chemmove[-stealth,shorten <=2pt,shorten >=2pt,dash pattern=on 1pt off 1pt,thin]{
        \draw([yshift=-4pt]atoo1.270) ..controls +(0:5mm) and +(right:10mm)..(sb1);
        \draw(sb2) ..controls +(up:10mm) and +(north west:10mm)..(atoc);
        \draw(db) ..controls +(right:5mm) and +(east:5mm)..(atoo2);}
      \arrow(@start.base west--){0}[-75,2]
      {}
      \arrow
      \chemfig{[:-30]*6(=-=(-C([1]-@{atoo2}O-[@{sb}0]@{atoh}H)([6]=O))-=-)}
      \arrow{0}
      \chemfig{[:-30]*6((-C(-[5]H)(-[7]H)-[2]@{atoo1}\chemabove{O}{\scriptstyle\ominus})-=-=-=)}
      \chemmove[-stealth,shorten >=2pt,dash pattern=on 1pt off 1pt,thin]{
        \draw[shorten <=7pt](atoo1.90) ..controls +(+90:8mm) and +(up:10mm)..(atoh);
        \draw[shorten <=2pt](sb) ..controls +(up:5mm) and +(up:5mm)..(atoo2);}
    \schemestop
  \end{center}
\end{scheme}
\end{VerbatimOut}

\MyIO


%\begin{scheme}[ht]
%  \caption{This is the third scheme caption.}
%  \vspace*{6pt}
%  \begin{center}
%    \schemestart
%      \chemfig{*6(=-=(-(=[2]O)-[%-60]O-[0]O-[%30](=[2]O)-[%-60]*6(=-=-=-))=-)}
%      \arrow{->[$\Delta$]}
%      2 \chemfig{*6(=-=(-(=[2]O)-[%-60]{0.,0})-=-)}
%      \arrow
%      2 \chemfig{*6(=-=(-[,.15,,,draw-none]{0.,})-=-)}\+\ch{2 CO2 ^}
%    \schemestop
%  \end{center}
%\end{scheme}
