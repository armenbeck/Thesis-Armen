\ProvidesFile{ap-mathematics.tex}[2021-08-23 mathematics appendix]

\begin{VerbatimOut}{z.out}
\chapter{MATHEMATICS}
\ix{mathematics//Mathematics appendix}

\PurdueThesisLogo\ loads the \AMSmathLogo\ package \cite{amslatex3project2019}
to do mathematics.
\end{VerbatimOut}

\MyIO


\begin{VerbatimOut}{z.out}
There are two types of mathematics in \LaTeX.
Text math is math that that is interspersed with text.
For example, this is text math: \(a = b + c\).
This is display math:
\begin{equation}
  a = b + c
\end{equation}
\end{VerbatimOut}

\MyIO


\begin{VerbatimOut}{z.out}
\newpage

\section{Standard Functions}

Standard functions should be in a roman font.
Like this: \(\cos\theta\).
Here is a list of standard function commands:\\

% The "@{\hspace*{\parindent}}" indents the table
% the same amount as a paragraph.
\begin{tabular}{@{\hspace*{\parindent}}llll@{}}
  \verb+\arccos+& \verb+\csc+& \verb+\ker+&    \verb+\min+\\
  \verb+\arcsin+& \verb+\deg+& \verb+\lg+&     \verb+\Pr+\\
  \verb+\arctan+& \verb+\det+& \verb+\lim+&    \verb+\sec+\\
  \verb+\arg+&    \verb+\dim+& \verb+\liminf+& \verb+\sin+\\
  \verb+\cos+&    \verb+\exp+& \verb+\limsup+& \verb+\sinh+\\
  \verb+\cosh+&   \verb+\gcd+& \verb+\ln+&     \verb+\sup+\\
  \verb+\cot+&    \verb+\hom+& \verb+\log+&    \verb+\tan+\\
  \verb+\coth+&   \verb+\inf+& \verb+\max+&    \verb+\tanh+\\
\end{tabular}
\ix
{%
  arccos//arcsin//arctan//arg//cos//cosh//cot//coth%
  //csc//deg//det//dim//exp//gcd//hom//inf%
  //ker//lg//lim//liminf//limsup//ln//log//max%
  //min//Pr//sec//sin//sinh//sup//tan//tanh%
}
\index{\verb+\arccos+} \index{\verb+\arcsin+} \index{\verb+\arctan+} \index{\verb+\arg+}
\index{\verb+\cos+} \index{\verb+\cosh+} \index{\verb+\cot+} \index{\verb+\coth+}
\index{\verb+\csc+} \index{\verb+\deg+} \index{\verb+\det+} \index{\verb+\dim+}
\index{\verb+\exp+} \index{\verb+\gcd+} \index{\verb+\hom+} \index{\verb+\inf+}
\index{\verb+\ker+} \index{\verb+\lg+} \index{\verb+\lim+} \index{\verb+\liminf+}
\index{\verb+\limsup+} \index{\verb+\ln+} \index{\verb+\log+} \index{\verb+\max+}
\index{\verb+\min+} \index{\verb+\Pr+} \index{\verb+\sec+} \index{\verb+\sin+}
\index{\verb+\sinh+} \index{\verb+\sup+} \index{\verb+\tan+} \index{\verb+\tanh+}
\end{VerbatimOut}

\MyIO


\begin{VerbatimOut}{z.out}
\newpage

\section{English Words in Math}

English words in math should be in a roman font like this:\\
Let the maximum value of \(a\) be \(a_\text{max}\).\\
\(a_\text{max} \ge a_\text{min}\) should always be true.\\
The temperature in the attic is \(t_\text{attic}\).
\end{VerbatimOut}

\MyIO


\begin{VerbatimOut}{z.out}
\section{Text Math}

Use \verb+\(+ to start text math and \verb+\)+ to end text math.
Some people use \verb+$+ to start and end text math---I don't
recommend that because \LaTeX\ can give better error messages
if you use \verb+\(+ and \verb+\)+.
\end{VerbatimOut}

\MyIO


\begin{VerbatimOut}{z.out}
\section{Displayed Equations}

Do not use \verb+$$+ to start or end displayed math like \TeX\ uses
\cite{gratzer2016}.

The \AMSmathLogo\ package provides a number
of additional displayed equation structures
beyond the ones provided in basic \LaTeX.
The augmented set includes \cite{amslatex3project2019b}:

\hbox to\hsize{%
  \hss
  \begin{tabular}{@{}ll@{}}
    \toprule
    \bfseries Environment& \bfseries Used for\\
    \midrule
    \tt equation& used for single equations\\
    \tt multline& split single equations over multiple lines\\
    \tt gather& collect but do not align multiple equations\\
    \tt align& align multiple equations\\
    \tt alignat& aligns multiple equations at multiple places\\
    \tt flalign& aligns multiple equations at multiple places on full length lines\\
    \tt split& split a single equation over multiple lines\\
    \bottomrule
  \end{tabular}%
  \hss
}

All but \verb+split+ can be followed by \verb+*+ to not number equations.
\end{VerbatimOut}

\MyIO


\begin{VerbatimOut}{z.out}

\subsection{\texttt{equation} environment}

The \verb+equation+ environment is used for single equations.

\begin{equation}
  E = mc^2
\end{equation}
\end{VerbatimOut}

\MyIO


\begin{VerbatimOut}{z.out}

The \verb+equation*+ environment does single, unnumbered equations.

\begin{equation*}
  a = b_0c + \frac12 de^2 + {\textstyle \frac12} fg^2
    + h_1 + h_2 + \cdots + h_n
    \qquad \text{for \(c \ne d\) and \(g < \infty\)}
\end{equation*}
\end{VerbatimOut}

\MyIO


\begin{VerbatimOut}{z.out}

\textcite{greene-2021-03-14} wrote
% \begin{quotation}  
  For
  \href{https://twitter.com/hashtag/PiDay?src=hashtag\_click\#PiDay}{\#PiDay},
  one of the coolest formulae for today's honoree:
  \[
    \frac 1\pi
    =
    \frac {\sqrt8} {9801}
    \sum_{n=0}^\infty
    \frac  {(4n!) (1103+26390n)}  {(n!)^4 396^{4n}}
  \]
% \end{quotation}
\end{VerbatimOut}

\MyIO


\begin{VerbatimOut}{z.out}

International standard ISO 80000-2:2019 \cite{iso8000022019}
states that $\mit e$,~$\mit i$,~$\mit j$,
and $\itpi$ should appear as
$e$,~$i$,~$j$
and~$\pi$ because they are constants.
This is done automatically by the pa-mismath package
that is loaded by thesis.tex.
See thesis.tex for more information,
including what to do if you're not using those as constants.

Euler's identity is
\begin{equation*}
  e^{i\pi} + 1 = 0.
\end{equation*}
\end{VerbatimOut}

\MyIO


\begin{VerbatimOut}{z.out}

Here's a simple formula relating $e$,~$i$,~$\pi$, and~$\phi$,
the golden ratio
\begin{equation}
  e^{i\pi} + 2\phi = \sqrt 5.
\end{equation}
I didn't notice anything on the web about putting the symbol for
the golden ratio in a special font even though it is a constant.
\end{VerbatimOut}

\MyIO


\begin{VerbatimOut}{z.out}

International standard ISO 80000-2:2019 \cite{iso8000022019}
states that the ``$d\/$'' in math differentials
should be typeset as ``$\di$''.
So,
\begin{equation*}
  \text{use } \int x\di x\qquad\qquad \text{instead of } \int x\,dx
\end{equation*}
\end{VerbatimOut}

\MyIO


\begin{VerbatimOut}{z.out}

The formula for Bekenstein-Hawking entropy:

\begin{equation*}
  S_\text{BH}
  =
  \frac A {4L_P^2}
  = \frac {c^3A} {4G\hbar}
\end{equation*}
\end{VerbatimOut}

\MyIO


\begin{VerbatimOut}{z.out}

Type in the math and let \LaTeX\ worry about the spacing.
You only need to do fine tuning by hand if it looks bad.

Another \verb+equation*+ environment,
note the spacing before the large close parenthesis:

\begin{equation*}
  \frac ab
    = ab^{-1}
    % Parens are the wrong size.
    = (\sqrt{\frac ab})^2
    % Parens are the right size but closing paren is too close to radical.
    = \left( \sqrt\frac ab \right)^2
    % Parens are right size but a negative thin space puts closing paren on top of radical.
    = \left( \sqrt\frac ab \!\right)^2
    % Parens are right size but a thin space puts closing paren too close to radical.
    = \left( \sqrt\frac ab \,\right)^2
    % Parens are right size but a medium space puts closing paren too close to radical.
    = \left( \sqrt\frac ab \:\right)^2
    % Parens are right size and I think a thick space looks the best.
    = \left( \sqrt\frac ab \;\right)^2
\end{equation*}
\end{VerbatimOut}

\MyIO


\begin{VerbatimOut}{z.out}

\begin{equation*}
  (\cos x)^2 + (\sin x)^2 = \cos^2 x + \sin^2 x = 1
\end{equation*}
\end{VerbatimOut}

\MyIO


\begin{VerbatimOut}{z.out}

\begin{equation}
  x \mod 2 =
  \begin{cases}
    0& \text{if $x$ is even}\\
    1& \text{if $x$ is odd}\\
  \end{cases}
\end{equation}
\end{VerbatimOut}

\MyIO


\begin{VerbatimOut}{z.out}

The first six derivatives of distance are velocity, acceleration, jerk, snap, crackle,
and pop \cite{reid2013}.

\begin{equation}
  % Every array element should be in \displaystyle (a big font).
  \AtBeginEnvironment{array}{\everymath{\displaystyle}}
  % Set space between columns to zero, use {} = ... to add a little space before the = "by hand".
  \arraycolsep = 0pt
  \text{distance derivitives} = \left\{\ %
    \begin{array}{llllllll}
      % I'm formatting the first 4 lines different from the last 3 so this will fit on one page.
      x&      {}=\text{distance}&     {}=vt\\[2pt]
      v&      {}=\text{velocity}&     {}=\frac{\di x}{\di t}\\[9pt]
      a&      {}=\text{acceleration}& {}=\frac{\di v}{\di t}& {}=\frac{\di^2x}{\di t^2}\\[9pt]
      \mit j& {}=\text{jerk}&         {}=\frac{\di a}{\di t}& {}=\frac{\di^2v}{\di t^2}&
        {}=\frac{\di^3x}{\di t^3}\\[9pt]
      s
        & {}=\text{snap}
        & {}=\frac{\di \mit j}{\di t}
        & {}=\frac{\di^2a}{\di t^2}
        & {}=\frac{\di^3v}{\di t^3}
        & {}=\frac{\di^4x}{\di t^4}\\[9pt]
      c
        & {}=\text{crackle}
        & {}=\frac{\di s}{\di t}
        & {}=\frac{\di^2\mit j}{\di t^2}
        & {}=\frac{\di^3a}{\di t^3}
        & {}=\frac{\di^4v}{\di t^4}
        & {}=\frac{\di^5x}{\di t^5}\\[9pt]
      p
        & {}=\text{pop}
        & {}=\frac{\di c}{\di t}
        & {}=\frac{\di^2s}{\di t^2}
        & {}=\frac{\di^3\mit j}{\di t^3}
        & {}=\frac{\di^4a}{\di t^4}
        & {}=\frac{\di^5v}{\di t^5}
        & {}=\frac{\di^6x}{\di t^6}
    \end{array}
  \right.
\end{equation}
\end{VerbatimOut}

\MyIO


\begin{VerbatimOut}{z.out}

\subsection{\texttt{multline} environment}

The \verb+multline+ environment is used
to split single equations over multiple lines.

\begin{multline}
  S = a + b + c + d + e + f + g + h + i + j\\
  + k + l + m + n + o + p\\
  + q + r + s + t + u + v + w + x + y + z
\end{multline}
\end{VerbatimOut}

\MyIO


\begin{VerbatimOut}{z.out}

\begin{multline}
  S = a + b + c + d + e\\
  + f + g + h + i + j\\
  + k + l + m + n + o\\
  + p + q + r + s + t\\
  + u + v + w + x + y\\
  + z
\end{multline}
\end{VerbatimOut}

\MyIO


\begin{VerbatimOut}{z.out}

% Calculate width of space before equation plus equation number.
% (All digits are the same width.)
\newdimen{\tdimen}
\settowidth{\tdimen}{\kern\multlinetaggap (L.5)}
\begin{multline}
  S = a + b + c + d + e\\
  \makebox[\textwidth]{\hfill $+ f + g + h + i + j$\hfill\hfill\hfill\hfill\kern\tdimen}\\
  \makebox[\textwidth]{\hfill\hfill${} + k + l + m + n + o$\hfill\hfill\hfill\kern\tdimen}\\
  \makebox[\textwidth]{\hfill\hfill\hfill${} + p + q + r + s + t$\hfill\hfill\kern\tdimen}\\
  \makebox[\textwidth]{\hfill\hfill\hfill\hfill${} + u + v + w + x + y$\hfill\kern\tdimen}\\
  + z
\end{multline}
\end{VerbatimOut}

\MyIO


\begin{VerbatimOut}{z.out}

\subsection{\texttt{gather} environment}

The \verb+gather+ environment collects but does not align multiple equations.

\begin{gather}
  a = b + c + d + e + f + g + h + i + j + k + l\\
  m = n + o + p + q + r + s + t + u + v + w + x + y + z
\end{gather}
\end{VerbatimOut}

\MyIO


\begin{VerbatimOut}{z.out}

\begin{gather}
  a = b + c + d + e + f + g + h + i + j + k + l\notag\\
  m = n + o + p + q + r + s + t + u + v + w + x + y + z
\end{gather}
\end{VerbatimOut}

\MyIO


\begin{VerbatimOut}{z.out}

\begin{gather*}
  \alpha = \beta + \gamma + \delta + \eta\\
  \theta = \iota + \kappa + \lambda + \mu + \nu + \rho + \tau
\end{gather*}
\end{VerbatimOut}

\MyIO


\begin{VerbatimOut}{z.out}

\begin{gather}
  x_\text{min} + x_\text{max} \le \sum_{i=1}^n x_i\\
  x_\text{min} + x_\text{max}
    = \sum_{i=1}^n x_i - \sum_{i=2}^{n-1} x_i\quad\text{if $x$ is sorted}\\
  x_\text{min} \le \left(\sum_{i=1}^n x_i\right) / n
\end{gather}
\end{VerbatimOut}

\MyIO


\begin{VerbatimOut}{z.out}

\subsection{\texttt{align} environment}

The \verb+align+ environment aligns multiple equations.

\begin{align}
  a &= b + c + d\\
  e &= f + g + h + i + j
\end{align}
\end{VerbatimOut}

\MyIO


\begin{VerbatimOut}{z.out}

\begin{align}
  x = \frac{-b \pm \sqrt{b^2-4ac}}{2a}\notag\\
  % Put a thin space before the b^2 to improve the appearance.
  x = \frac{-b \pm \sqrt{\,b^2-4ac}}{2a}
\end{align}
\end{VerbatimOut}
\ix{align environment}
\index{\verb+\begin{align}+}
\ix{thin space}
\index{\verb+\,+}

\MyIO


\begin{VerbatimOut}{z.out}

Quadratic formula proof \cite{khan}:
\ix{quadratic formula}

% The align environment requires the amsmath package.
% Use \addtolength{\jot}{6pt} to increase the space between rows in an amsmath multi-line math formula.
% That's not done here so everything will fit on one page.
\begin{align}
  ax^2 + bx + c &= 0\\
  ax^2 + bx &= -c\notag\\
  % The "\," adds a thinspace of horizontal space.
  x^2 + \frac ba\,x &= -\frac ca\notag\\
  x^2 + \frac ba\,x + \frac{b^2}{4a^2} &= \frac{b^2}{4a^2} - \frac ca\notag\\
  \left(x + \frac b{2a}\right)^2 &= \frac{b^2}{4a^2} - \frac ca\notag\\
  \left(x + \frac b{2a}\right)^2 &= \frac{b^2}{4a^2} - \frac{4ac}{4a^2}\notag\\
  \left(x + \frac b{2a}\right)^2 &= \frac{b^2-4ac}{4a^2}\notag\\
  \sqrt{\left(x + \frac b{2a}\right)^2}
    &= \sqrt{\left(\frac{b^2-4ac}{4a^2}\right)}\notag\\
  x + \frac b{2a} &= \pm \frac{\sqrt{\,b^2-4ac}}{\sqrt{4a^2}}\notag\\
  x + \frac b{2a} &= \pm \frac{\sqrt{\,b^2-4ac}}{2a}\notag\\
  x &= - \frac b{2a} \pm \frac{\sqrt{\,b^2-4ac}}{2a}\notag\\
  x &= \frac{-b \pm \sqrt{\,b^2-4ac}}{2a}
\end{align}
\end{VerbatimOut}

\MyIO


\begin{VerbatimOut}{z.out}

\subsection{\texttt{alignat} environment}
\index{\verb+\begin{aligo}+@\verb+\begin{alignat}+}
\ix{alignat environment}

The \verb+alignat+ environment aligns multiple equations at multiple places.
\begin{alignat}{2}
  a &= b& \qquad\qquad& \text{set $a$}\\
  c &= d& &             \text{you guessed it, set $c$}\notag\\
  g &= h& &             \text{and finally, set $g$}
\end{alignat}
\index{\verb+\begin{aligo}+@\verb+\begin{alignat}+}
\ix{alignat environment}
  
I like to align input columns on the input if possible
and will sometimes use windows over~250 characters wide to do so.
If that won't work I sometimes do,
for example,
\begin{alignat}{2}
  a
    &= b
    & \qquad\qquad
    & \text{set $a$}\\
  c
    &= d
    &
    &\text{you guessed it, set $c$}\notag\\
  g
    &= h
    &
    &\text{and finally, set $g$}
\end{alignat}
\index{\verb+\begin{aligo}+@\verb+\begin{alignat}+}
\ix{alignat environment}

Do whatever works best for you.

\end{VerbatimOut}

\MyIO


\begin{VerbatimOut}{z.out}

Quadratic formula proof \cite{khan}:

% Make changes to \jot be local to the group that starts on the next line.
{
  % Increase distance between lines by 6pt.
  \addtolength{\jot}{6pt}
  \begin{alignat}{2}
    ax^2 + bx + c
      &= 0
      &
      &\text{subtract $c$}\\
    ax^2 + bx
      &= -c
      &
      &\text{divide by $a$}\notag\\
    % The "\," adds a thinspace of horizontal space.
    x^2 + \frac ba\,x
      &= -\frac ca
      &
      &\text{add $\displaystyle\frac{b^2}{4a^2}$}\notag\\
    x^2+\frac ba\,x+\frac{b^2}{4a^2}
      &= \frac{b^2}{4a^2}-\frac ca
      &
      &\text{factor left hand side}\notag\\
    \left(x+\frac b{2a}\right)^2
      &= \frac{b^2}{4a^2}-\frac ca
      &
      &\text{multiply right-most term by $\displaystyle\frac{4a}{4a}$}\notag\\
    \left(x + \frac b{2a}\right)^2
      &= \frac{b^2}{4a^2}-\frac{4ac}{4a^2}
      &
      &\text{use common denominator}\notag\\
    \left(x + \frac b{2a}\right)^2
      &= \frac{b^2-4ac}{4a^2}
      &
      &\text{take square root of each side}\notag\\
    \sqrt{\left(x + \frac b{2a}\right)^2}
      &= \sqrt{\left(\frac{b^2-4ac}{4a^2}\right)}
      &
      &\text{compute square root of each side}\notag\\
    x + \frac b{2a}
      &= \pm \frac{\sqrt{\,b^2-4ac}}{\sqrt{4a^2}}
      &
      &\text{simplify right hand denominator}\notag\\
    x + \frac b{2a}
      &= \pm \frac{\sqrt{\,b^2-4ac}}{2a}
      &
      &\text{subtract $\displaystyle\frac b{2a}$ from each side}\notag\\
    x
      &= -\frac b{2a} \pm \frac{\sqrt{\,b^2-4ac}}{2a}
      &\qquad
      &\text{use common denominator}\notag\\
    x
      &= \frac{-b \pm \sqrt{\,b^2-4ac}}{2a}
  \end{alignat}
}
\end{VerbatimOut}

\MyIO


\begin{VerbatimOut}{z.out}

\index{\verb+\begin{flalign}+}
\todoindex{Verb+Begin-Ocurly-flalign-Ccurly+}
\ix{falign environment}
\subsection{\texttt{flalign} environment}

The \verb+flalign+ environment aligns multiple equations at multiple places
on full length lines.

\begin{flalign}
  a &= b&   &   & u &= v\\
  c &= d& m &= n& w &= x\notag\\
  g &= h&   &   & y &= z
\end{flalign}
\end{VerbatimOut}

\MyIO


\begin{VerbatimOut}{z.out}

\index{\verb+\begin{split}+}
\todoindex{Verb+Begin-Ocurly-split-Ccurly+}
\ix{split environment}
\subsection{\texttt{split} environment}

The \verb+split+ environment ???.
\end{VerbatimOut}

\MyIO
\index{\verb+\begin{split}+}
\todoindex{Verb+Begin-Ocurly-split-Ccurly+}
\ix{split environment}


\begin{VerbatimOut}{z.out}
\section{Theorem-like environments}

These theorem-like environments are defined
in the amsthm package or in\\  % break line here so we don't go past right margin
\verb+PurdueThesis.cls+.
\end{VerbatimOut}

\MyIO


\begin{VerbatimOut}{z.out}

\index{\verb+\begin{definition}+}
\todoindex{Verb+Begin-Ocurly-definition-Ccurly+}
\ix{definition environment}
\begin{definition}
  This is an example definition.
\end{definition}

\index{\verb+\begin{observation}+}
\todoindex{Verb+Begin-Ocurly-observation-Ccurly+}
\ix{observation environment}
\begin{observation}
  This is an example observation.
\end{observation}

\index{\verb+\begin{proof}+}
\todoindex{Verb+Begin-Ocurly-proof-Ccurly+}
\ix{proof environment}
\begin{proof}
  This is an example proof.
  If \(a = b\) and \(b = c\) then \(a = c\).
\end{proof}

\index{\verb+\begin{proposition}+}
\todoindex{Verb+Begin-Ocurly-proposition-Ccurly+}
\ix{proposition environment}
\begin{proposition}
  This is an example proposition.
\end{proposition}

\index{\verb+\begin{theorem}+}
\todoindex{Verb+Begin-Ocurly-theorem-Ccurly+}
\ix{theorem environment}
\begin{theorem}
  This is an example theorem.
\end{theorem}
\end{VerbatimOut}

\MyIO


\begin{VerbatimOut}{z.out}


\section{Examples}

\subsection{Bayes' Theorem}
\ix{Bayes' Theorem}

Bayes' Theorem \cite{bayes}:
{
  \UndefineShortVerb{\|}
\[
  \text{P}(\text A|\text B)
    % The "\," puts a thin horizontal space there, 1/6 of an "em".
    % An "em" is roughly the width of a lowercase "m".
    = \frac{\text P(\text B|\text A)\,\text P(\text A)}{\text P(\text B)}
\]
}
\end{VerbatimOut}

\MyIO


\begin{VerbatimOut}{z.out}

\subsection{Nicomachus's theorem}
\ix{Nicomachus's theorem}

Nicomachus's theorem \cite{wikipedia-nicomachus} states that
the sum of the first~$n$ cubes is the square of the~$n$th triangular number.
That is,
\[
  1^3 + 2^3 + 3^3 + \cdots + n^3 = (1 + 2 + 3 + \cdots + n)^2.
\]
The same equation may be written more compactly using the mathematical notation for summation:
\[
  \sum_{k=1}^n k^3 = \left(\sum_{k=1}^n k\right)^2.
\]
Also see the diagram on that web page.
\end{VerbatimOut}

\MyIO


\begin{VerbatimOut}{z.out}

\subsection{Prime Number Theorem}
\ix{Prime Number Theorem}

\textcite{li2013} suggested using a functional equation
from the Prime Number Theorem proof
as an example:
\begin{equation}
  \int_1^x
    \sum_{p\le u}
    \left\lfloor\frac{\log u}{\log p}\right\rfloor
    \log p
    \,\text{d}u
    =
    \frac1{2\pi i}
    \int_{c-i\infty}^{c+i\infty}
    \frac{x^{s+1}}{s(s+1)}
    \left(-\frac{\zeta'(s)}{\zeta(s)}\right)
    \text{d}s
\end{equation}
\end{VerbatimOut}

\MyIO


\begin{VerbatimOut}{z.out}

\subsection{Quantum Mechanics}
\ix{Quantum Mechanics}

\textcite{greene-2021-04-04} wrote
\ix{Greene, Brian Randolph}
\begin{quotation}
  Quantum Mechanics in a nutshell:
  A particle goes from here to there
  by sampling every possible trajectory from here to there.

  \[
    \langle x_f,t_f \vert x,t_{\mathit i} \rangle
    =
    \sum_{\text p \in \text{paths}} e^{\mathit iS(\text p) \si{\planckbar}}
  \]
\end{quotation}
\end{VerbatimOut}

\MyIO


%  https://tex.stackexchange.com/questions/96568/how-can-i-align-multiple-cases-environment-simultaneously
\begin{VerbatimOut}{z.out}
\subsection{Question in String Theory / Mass of States / Number Operator}


\textcite{yourlazyphysicist2017} wrote
``I have the following definition of the space-time coordinates'':

\newcommand{\fpt}{{4\pi T}}
\newcommand{\oh}{\frac12}
\newcommand{\snnz}{\sum_{n\ne0}}
\newcommand{\tms}{\tau - \sigma}
\newcommand{\tps}{\tau + \sigma}
\begin{align}
    \text{closed string: }&
        \begin{cases}
            \displaystyle
            X^\mu_R
                = \oh x^\mu
                + \frac1\fpt (\tms) p^\mu
                + \frac i{\sqrt\fpt} \snnz \frac1n \alpha^\mu_n e^{-in(\tms)},\\
            \displaystyle
            X^\mu_L
                = \oh x^\mu
                + \frac1\fpt (\tps) p^\mu
                + \frac i{\sqrt\fpt} \snnz \frac1n \tilde\alpha^\mu_n e^{-in(\tps)}.
        \end{cases}\\[6pt]
    \text{open string: }&
        \begin{cases}
            \displaystyle
            X^\mu_N
                = x^\mu
                + \frac1{\pi T}p^\mu\tau
                + \frac i{\sqrt{\pi T}} \snnz \frac1n \alpha^\mu_n e^{-in\tau} \cos(n\sigma),\\
            \displaystyle
            X^\mu_D
                = x^\mu
                + \frac i{\sqrt{\pi T}} \snnz \frac1n \alpha^\mu_n e^{-in\tau} \sin(n\sigma).
        \end{cases}
\end{align}
\end{VerbatimOut}

\MyIO



% https://www.google.com/url?sa=i&rct=j&q=&esrc=s&source=images&cd=&cad=rja&uact=8&ved=0ahUKEwiSzf370-jmAhXIGs0KHRHyAZQQMwiCASgNMA0&url=https%3A%2F%2Fwww.chegg.com%2Fhomework-help%2Fquestions-and-answers%2Fequations-displacement-atoms-along-linear-chain-d2u-u-m-mass-atoms-c-force-constant-neares-q26052186&psig=AOvVaw1eYq-Q0El0Tbjgbp_Lu2Vv&ust=1578183004388936&ictx=3&uact=3

% M\frac{d^2\mu_n}{dt^2} = C(u_{n_1} - 2u_n + u_{n-1}).

% a^{b} same as a^b     c^{de} f^{g^h} ij^{kl}^{mn}

% superscripts
% subscripts

% operators

%\sum
%\prod

%\alpha
%\beta
%\gamma
%\delta
%\epsilor
%\varepsilon
%\zero
%\eta
%\theta
%\vartheta
%\iota
%\kappa
%\lambda
%\mu
%\nu\xi
%\varrho
%\sigma
%\varsigma
%\tau
%\upsilan
%\pha
%\varphi
%\chi
%\psi
%\omega

%\Gamma
%\Delta
%\Theta
%\Lambda
%\Xi
%\Pi
%\Sigmo
%\Upsilan
%\Phi
%\Psi
%\Omega

%mixed greek and text

%x_min, x_max

%\ldots

%\to

%1 + \frac14 + \frac19 + \cdots = \frac\pi6

%fractions

% frac{\ln x}{\ln\alpha} = log_\alpha x

% \sum_{i=1}^n = 1 + 2 + \cdots + n = \frac{n(n+1)}{2}


% xxxx     \end{verbatim}
% xxxx % Requires \usepackage{amsmath}; use align* for no equation number.
% xxxx \begin{align}
% xxxx   a = {}& b + c\\
% xxxx   x = {}& y + z
% xxxx \end{align}
% xxxx     \vskip\baselineskip
% xxxx     \hrule
% xxxx     \vskip0.5\baselineskip
% xxxx     \filbreak
% xxxx
% xxxx     \begin{verbatim}
% xxxx \[
% xxxx   Z =
% xxxx     \left(
% xxxx       \begin{array}{cc}
% xxxx         a& b\\
% xxxx         c& d
% xxxx       \end{array}
% xxxx     \right)
% xxxx \]
% xxxx     \end{verbatim}
% xxxx \[
% xxxx   Z =
% xxxx     \left(
% xxxx       \begin{array}{cc}
% xxxx         a& b\\
% xxxx         c& d
% xxxx       \end{array}
% xxxx     \right)
% xxxx \]
% xxxx     \vskip\baselineskip
% xxxx     \hrule
% xxxx     \vskip0.5\baselineskip
% xxxx     \filbreak
% xxxx
% xxxx     \begin{verbatim}
% xxxx \begin{equation}
% xxxx   \begin{split}
% xxxx     a = {}& b + c\\
% xxxx       & {} + d + e
% xxxx   \end{split}
% xxxx \end{equation}
% xxxx     \end{verbatim}
% xxxx \begin{equation}
% xxxx   \begin{split}
% xxxx     a = {}& b + c\\
% xxxx       & {} + d + e
% xxxx     \end{split}
% xxxx \end{equation}
% xxxx     \vskip\baselineskip
% xxxx     \hrule
% xxxx     \vskip0.5\baselineskip
% xxxx     \filbreak
% xxxx
% xxxx     \begin{verbatim}
% xxxx \[
% xxxx   (\cos x)^2 + (\sin x)^2 = 1
% xxxx \]
% xxxx     \end{verbatim}
% xxxx \[
% xxxx   (\cos x)^2 + (\sin x)^2 = 1
% xxxx \]
% xxxx     \vskip\baselineskip
% xxxx     \hrule
% xxxx     \vskip0.5\baselineskip
% xxxx     \filbreak
% xxxx
% xxxx     \begin{verbatim}
% xxxx If $X = \cos x$ and $Y = \sin x$ then $X^2 + Y^2 = 1$.
% xxxx     \end{verbatim}
% xxxx If $X = \cos x$ and $Y = \sin x$ then $X^2 + Y^2 = 1$.
% xxxx     \vskip\baselineskip
% xxxx     \hrule
% xxxx     \vskip0.5\baselineskip
% xxxx     \filbreak
% xxxx
% xxxx
