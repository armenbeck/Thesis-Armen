\ProvidesFile{ch-introduction.tex}[2021-08-23 introduction chapter]

\chapter{INTRODUCTION}


{\sl\TeX\/} is a typesetting system for the creation of beautiful books---%
and especially for books that contain lots of mathematics
\cite[page v]{knuth2012}.

{\sl\LaTeX\/} is a software system for typesetting documents
\cite[back cover]{lamport1994}.
It extends \TeX\ with more natural chapter,
section,
etc.~commands that are easier to use.
\LaTeX\ has document classes for
articles,
books,
reports,
etc.

{\sl\PurdueThesisLogo\/}
({\sl\PuThLogo} for short---rhymes with tooth)
is a \LaTeX\ document class used for Purdue theses,
dissertations,
master’s bypass reports,
and PhD preliminary reports.
This template demonstrates how to use PurdueThesis.
PurdueThesis supports all Purdue campuses,
programs,
and graduate degrees.

The Thesis and Dissertation Office wrote a manual \cite{thesis2017}
and Microsoft Word templates \cite{thesis2020}.

Draizelle Sexon \cite{sexon2012}
\todoerror{%
  'D. Sexon.  ``The thesis.'' (Sep.~18,~2012), [Online].\\
  Available:~https://www.slideshare.net/draizelle\_sexon\\
  /the-thesis-and-its-parts.'
  gets printed for this reference.
  The URL contains a \_,
  the URL is invisible in the bibliography
  but copy/paste shows it.%
}
recommends using these chapter names:\\
\I2 Problem and Its Background\\
\I2 Review of Related Literature and Studies\\
\I2 Methodology of the Study\\
\I2 Presentation, Analysis and Interpretation of Data\\
\I2 Summary, Conclusions, and Recommendations

Mantian Xue's \cite{xue2019} thesis contained these chapters:\\
\I2 Introduction\\
\I2 Device Technology\\
\I2 Graphene-based Biosensors\\
\I2 Graphene-based Ion Sensing\\
\I2 MoS${}_2$-based Sensors\\
\I2 Conclusion and Future Work

Think about the structure of your thesis
and use appropriate chapter names.


\section{Typographic Conventions}

{
  \newlength{\Parindent}
  \setlength{\Parindent}{\parindent}
  \newcommand{\Indent}{\advance\leftskip by \Parindent}
  \parindent = 0pt

  \newcommand{\Describe}[2]{%
    {\Indent \Indent #1\endgraf}%
    {\Indent \Indent \Indent #2\endgraf}%
  }

  {%
    \Indent The following typographic conventions
    are used in this document.
    These conventions were influenced by
    \cite{wireshark-users-guide,dijk2000,weh2016}%
    \todowarn{Change to, for example, [10--12] here and everywhere.}.
    There are no quotes in the typographic conventions.\endgraf
  }
  
  % Admonitions information is in \cite{wireshark-users-guide}.
  % Admonitions are not supported yet.

  % Dialog and window buttons information is in \cite{wireshark-users-guide}.
  % Dialog and window buttons are not supported yet.

  \Describe
  {\Emph{Emphasis}, \First{First Use}, and \Title{Title}}
  {%
    Emphasis: You \Emph{must} do this.

    First Use: The sensor was installed in an \First{ekayak}.
    An ekayak is an electric kayak.

    Title: He read \Title{The Grapes of Wrath}
    and watched \Title{Citizen Kane}.%
  }

  \Describe
  {\Keys{Keyboard} \Keys{Keys}}
  {%
    \Keys{Control + A} means press the Control key and A key at the same time.
    \Keys{A} \Keys{B} means press key A and then press key B.%
  }

  % Can't use \verb iside an arguement to \Describe.
  \Describe
  {{\tt Literal Elements}}
  {%
    Literal elements include checkboxes,
    code,
    environment variables,
    file names,
    function names,
    \LaTeX\ input,
    output,
    variable names,
    and verbatim input (except for commands typed on the command line).
    {\tt\char'40} is used to indicate a space
    if it is not clear where spaces are.
  }
  \index{.@"\verb*+ + (visible space)}
  \index{"\verb+"\char'40+}

  \Describe
  {\Menu{Menu > Item}}
  {%
    To make sure smooth scrolling is on go to
    \Menu{Edit > Settings}
    and make sure the
    {\tt Use smooth scrolling}
    checkbox is checked.%
  }

  \Describe
  {\Place{Placeholders}}
  {Placeholders need to be replaced with real input.}
  
  \Describe
  {\Shell{\ttfamily\bfseries shell commands}}
  {Commands typed on the command line by the user.}
  
}


\section{Writing in English Information}

\subsection{Logical punctation}

I use logical punctuation \cite{yagoda2011}:\\
  \I2 The sign said ``Buses Only''.\\
instead of\\
  \I2 The sign said ``Buses Only.''\\
so quoted material,
and only quoted material,
is inside quotes.
This is relatively new and not many people use it.
Your major professor may not like this style.
Check with them before you decide to use this.


\subsection{Serial comma}
\ix{, (comma)//comma//serial comma}

I use the serial comma:\\
  \I2 apple, berry, and cherry\\
instead of\\
  \I2 apple, berry and cherry\\
because I find it easier
to see the list items
when they are separated by commas.
The serial comma is also known as the
Oxford comma,
Harvard comma,
or series comma.

\section{\LaTeX-related information}

\subsection{Input reading rules}

\LaTeX\ uses the following rules when reading input:
\begin{itemize}
  \item the end of a line is equivalent to a space
  \item spaces at the beginning of a line are ignored
  \item a blank line ends a paragraph
\end{itemize}

% The itemize environment takes up lots of space---sometimes
% I like to compress the layuot as shown below.

\subsection{Input preparation conventions}

In \LaTeX\ typing

\begin{verbatim}
As \(h\) approaches 0 in the limit, the last fraction can be shown to go
 to zero.  This is true because the area of the red portion of excess re
gion is less than or equal to the area of the tiny black-bordered rectan
gle.  More precisely, \[\left|f(x)-\frac{A(x+h)-A(x)}h\right|-\frac{\lef
t|\text{Red Excess}\right|}h\le\frac{h\big(f(x+h_1)-f(x+h_2)\big)}h=f(x+
h_1)-f(x+h_2),\] where \(x+h_1\) and \(x+h_2\) are points where \(f\) re
aches its maximum and its minimum, respectively, in the interval \([x,x+
h]\).
\end{verbatim}

gives exactly the same output as

\begin{verbatim}
As \(h\) approaches 0 in the limit,
the last fraction can be shown to go to zero.
This is true because the area of the red portion of excess region
is less than or equal to the area of the tiny black-bordered rectangle.
More precisely,
\[
  \left |
    f(x)
    -
    \frac {A(x+h)-A(x)} {h}
  \right |
  -
  \frac {\left|\text{Red Excess}\right|} {h}
  \le
  \frac {h\big(f(x+h_1)-f(x+h_2)\big)} {h}
  =
  f(x+h_1)
  -
  f(x+h_2),
 \]
 where \(x+h_1\)
 and \(x+h_2\) are points where \(f\) reaches its maximum and its minimum,
 respectively,
 in the interval \([x, x + h]\).
\end{verbatim}
    

I've used \LaTeX\ over~30 years
and use these personal conventions
to prepare input.
Using these conventions leads
to many short lines,
but I find those easier
to read and edit.
Do whatever works best for you.

\I2 start input lines with\\
  \I3 the first word of a sentence\\
  \I3 \verb+(+\\
  \I3 \verb+and+\\
  \I3 \verb+but+\\
  \I3 \verb+from+\\
  \I3 \verb+or+\\
  \I3 \verb+to+

\NL
\I2 end input lines with\\
  \I3 sentence-ending periods\\
  \I3 phrase-ending commas\\
  \I3 phrase-ending colons\\
  \I3 phrase-ending semicolons\\
  \I3 \verb+)+\\
  \I3 \verb+\\+\\
  \I3 \verb+\\[+\textit{dimension}\verb+]+

\NL
\I2 put these on a line of their own\\
  \I3 \verb+\begin{+\textit{environment name}\verb+}+\\
  \I3 \verb+\end{+\textit{environment name}\verb+}+\\
  \I3 short parenthetical remark


\section{Filenames}
\ix{filenames}

There are several different name styles for file names:
\ix{camelCase//kebab-case//PascalCase//snake\_case}

{%
  \singlespace
  \I2
  \begin{tabular}{@{}ll@{}}
    \toprule
    \bf Name& \bf Why it's called that\\
    \midrule
    camelCase& |C| is taller that surrounding characters,
      looks like camel's hump\\
    kebab-case& letters appear to be slid on shish-kebab skewer,
      no \Keys{Shift} needed\\
    PascalCase& popular in the Pascal programming language\\
    snake\_case& looks like a snake, is kebab-case except
      |-| is changed to |_|\\
    \bottomrule                        
  \end{tabular}
}

\vspace*{6pt}
\textcolor{red}{I recommend you only use}
kebab-case file names that consist of only lowercase letters,
zero or more \verb+-+ characters
(but no consecutive \verb+-+ characters),
and a single period.
%%%%% You won't need to use \Keys{Shift} then.    2021-08-06

\textcolor{red}{Do not put spaces in your file names.}
It makes it easier to run your thesis on other computers.
  
I like
to start all chapter file names with |ch-|.
Chapter names are everything
from the beginning of the thesis through the last chapter.
Chapters include all front matter in addition
to all chapters.

Appendix names start with |ap-|
and are everything after the last chapter including any bibliography,
colophon,
indices,
and vita.

Graphics files specific to your thesis start
with \verb+gr-+ and go in the graphics folder.
Non-thesis graphics files retain their normal names
and go in the graphics folder.

\LaTeX\ package files specific to your thesis  start
with \verb+pa-+ and go in the packages folder.
Non-thesis packages retain their normal names
and go in the packages folder.


\section{Special input characters}

\UndefineShortVerb{\|}
\DefineShortVerb{\;}  % so ";verbatim;" will be verbatim
\noindent
{
  \setlength{\tabcolsep}{8.5pt}
\begin{tabular}{@{}lllllllllll@{}}
  \multicolumn{10}{@{}l}{These input characters are special:\hfil}\\
  \kern\parindent& ;#;&  ;$;&  ;%;&  ;&;&  ;\;&            ;^;&         ;_;&  ;{;&  ;};&  ;~;\\
  \multicolumn{10}{@{}l}{Type\hfil}\\
  \kern\parindent& ;\#;& ;\$;& ;\%;& ;\&;& ;$\backslash$;& ;\char'136;& ;\_;& ;\{;& ;\};& ;\char'176;\\
  \multicolumn{10}{@{}l}{to get this output\hfil}\\
  \kern\parindent& \#&   \$&   \%&   \&&   $\backslash$&   \char'136&   \_&   \{&   \}&   \char'176\\
\end{tabular}
}
\UndefineShortVerb{\;}
\DefineShortVerb{\|}  % so "|verbatim|" will be verbatim


\section{Spacing after periods}

% One or more \emph{lowercase}/\emph{uppercase}
% letters followed by a period is treated like the
% \emph{end of a sentence}/\emph{a person's middle initial}\\[6pt]
% with approximately
% \emph{two}/\emph{one}
% space(s) following the period.
% \ix{. (period)}

One or more
\lower6pt\hbox{\rlap{\small lowercase}}%
\raise6pt\hbox{\small uppercase}
letters followed by a period is treated like
\raise6pt\hbox{\rlap{\small a middle initial}}%
\lower6pt\hbox{\small the end of a sentence}\break

\vspace*{-12pt}
\noindent with approximately
\raise6pt\hbox{\small \rlap{one}}%
\lower6pt\hbox{\small two}
space(s) following the period.%
\ix{. (period)}

\begin{tabular}{@{}lll@{}}
  \bfseries Input& \bfseries Output& \bfseries Comment\\
  \noalign{\vspace{2pt}}
  \verb+Dr. Smith+& Dr. Smith& too much space after abbreviation\\
  \verb+Dr.\ Smith+& Dr.\ Smith& correct, Dr.\ and Smith can be on different lines\\
  \verb+Dr.~Smith+& Dr.~Smith& correct, Dr.\ and Smith will be on same line, I\\
  & & recommend using this\\
  \noalign{\vspace{2pt}}
  \verb+at NASA.  The+& at NASA.  The& not enough space after sentence ending period\\
  \verb+at NASA\@.  The+& at NASA\@.  The& correct\\
\end{tabular}
\index{Dr.~Smith}
\index{Smith}
\index{\verb+~+ (tilde)}
\index{NASA}
\index{\verb*+\ +}
\index{\verb+"\"@+}


\section{Four kinds of dashes}

There are four kinds of dashes
\ix{dash}

\begin{description}

  \item[hyphen]
  \ix{dash//hyphen//dash!hyphen}
    The hyphen
    is a punctuation mark used to join words
    and to separate syllables of a single word.
    \cite{wikipedia-hyphen}.
    
    \begin{singlespace}
      \begin{tabular}{@{}lll@{}}
        \toprule
        \bfseries Input& \bfseries Output& \bfseries Comment\\
        \midrule
        \verb+-+ (one hyphen)& -\\
        \verb+son-in-law+& son-in-law& used to join words\\
        \verb+gas-oline+& gas-oline& used to separate syllables, \LaTeX\ hyphenates words\\
        & & automatically so you may not ever use this\\
        \bottomrule
      \end{tabular}
    \end{singlespace}

  \item[endash]
  \ix{dash//endash//dash!endash}
    The endash
    \cite{wikipedia-endash}
     is used for
    \begin{singlespace}
      \begin{tabular}{@{}lll@{}}
        \toprule
        \bfseries Input& \bfseries Output& \bfseries Comment\\
        \midrule
        \verb+--+ (two hyphens)& --\\
        \verb+The Purdue--IU game+& The Purdue--IU game& conflict\\
        \verb+Perth--Dubai--Boston+& Perth--Dubai--Boston& connection\\
        \verb+Teal Road runs East--West+& Teal Road runs East--West& direction\\
        \verb+ages 21--65+& ages 21--65& age range\\
        \verb+June--July 1967+& June--July 1967& month range\\
        \verb+pages 38--55+& pages 38--55& page range\\
        \verb+1:15--2:15 p.m.+& 1:15--2:15 p.m.& time range\\
        \verb+Purdue beat IU 35--28+& Purdue beat IU 35--28& scores\\
        \bottomrule
      \end{tabular}
    \end{singlespace}

  \item[emdash]
  \ix{dash//emdash//dash!emdash}
    The emdash
    \cite{wikipedia-emdash}
     is used for
    \begin{singlespace}
      \begin{tabular}{@{}ll@{}}
        \toprule
        \verb+---+ (three hyphens)& ---\\[6pt]
        \bfseries Input& \verb+the usual suspects---Larry, Moe, and Curly+\\
        \bfseries Output& the usual suspects---Larry, Moe, and Curly\\
        \bfseries Comment& --- acts like colon\\[6pt]
        \bfseries Input& \verb+Larry, Moe, and Curly---the usual suspects+\\
        \bfseries Output& Larry, Moe, and Curly---the usual suspects\\
        \bfseries Comment& inverse function of colon\\[6pt]
        \bfseries Input& \verb+three people---Larry, Moe, and Curly---%+\\
        & \verb+are the usual suspects+\\
        \bfseries Output& three people---Larry, Moe, and Curly---are the usual suspects\\
        \bfseries Comment& first --- acts as \verb+(+, second --- acts as \verb+)+\\[6pt]
        \bfseries Input& \verb+I believe I shall---no, I'm going to do it.+\\
        \bfseries Output& I believe I shall---no, I'm going to do it.\\
        \bfseries Comment& use --- when a thought evolves on the fly\\
        \bottomrule
      \end{tabular}
    \end{singlespace}

    Emdashes should be used sparingly in formal writing.

  \item[figure dash]
  \ix{dash//figure dash//dash!figure dash}
    The figure dash
    (input: \verb+\FigureDash+)
    is used to separate digits---it's the same width
    as a digit and is used in identification numbers,
    part numbers,
    phone numbers, etc.
    Type, for example, \verb+Q6759\FigureDash 18100+
    to get ``Q6759\FigureDash 18100''.

  \item[minus sign]
  \ix{dash//minus sign//dash!minus sign}
    Used for negative numbers or subtraction in math mode.
    \begin{singlespace}
      \begin{tabular}{@{}lll@{}}
        \toprule
        \bfseries Input& \bfseries Output& \bfseries Comment\\
        \midrule
        \verb+-+& \(-\)& (one hyphen in text or display math mode)\\
        \verb=\(-a + b\)=& \(-a + b\)& negative $a$\\
        \verb+\(a - b\)+& \(a - b\)& subtraction\\
        \bottomrule
      \end{tabular}
    \end{singlespace}

\end{description}

% The first use of an unusual term is \First{emphasized}.
% It is defined soon after it is emphasized.
% From http://everets.org/kevin/ten-codes.php retrieved on 2020-02-28:
%     CODE  MEANS
%     10-7  Out of service
% So, the model number is a joke.
% 10\FigureDash 7
% for this research.
